\documentclass[pdftex,12pt, oneside]{article}

%\usepackage[paperwidth=8.5in, paperheight=13in]{geometry} % Folio
\usepackage[paperwidth=8.27in, paperheight=11.69in]{geometry} % A4

\usepackage{makeidx}         % allows index generation
\usepackage{graphicx}        % standard LaTeX graphics tool
                             % when including figure files
\usepackage[bottom]{footmisc}% places footnotes at page bottom
\usepackage[english]{babel}
\usepackage{enumerate}
\usepackage{paralist}
\usepackage{float}
\usepackage{gensymb}  
\usepackage{listings}
\usepackage{color}
\usepackage{mathtools} % atau \usepackage{amsmath}
\renewcommand{\baselinestretch}{1.5}

\newcommand{\HRule}{\rule{\linewidth}{0.5mm}}

\definecolor{codegreen}{rgb}{0,0.6,0}
\definecolor{codegray}{rgb}{0.5,0.5,0.5}
\definecolor{codepurple}{rgb}{0.58,0,0.82}
\definecolor{backcolor}{rgb}{0.95,0.95,0.92}

\lstdefinestyle{mystyle}{
  backgroundcolor=\color{backcolor},
  commentstyle=\color{codegreen},
  keywordstyle=\color{magenta},
  stringstyle=\color{codepurple},
  basicstyle=\footnotesize,
  breakatwhitespace=false,
  breaklines=true,
  captionpos=b,
  keepspaces=true,
  numbers=left,
  numbersep=5pt,
  showspaces=false,
  showstringspaces=false,
  showtabs=false,
  tabsize=2
}

\lstset{style=mystyle}


\begin{document}
\sloppy % biar section ga melebar melewati kertas

\begin{center}
{\large RANCANGAN SISTEM \textit{WEB SERVICES} SEBAGAI CARA KOMUNIKASI DENGAN TEMPAT PEMBAYARAN DALAM PENCATATAN PEMBAYARAN PAJAK BUMI DAN BANGUNAN PERDESAAN DAN PERKOTAAN DI KABUPATEN BREBES.}
\\[1cm]
DD MMM 2016\\
Priyanto Tamami, S.Kom.
\end{center}

%\frontmatter%%%%%%%%%%%%%%%%%%%%%%%%%%%%%%%%%%%%%%%%%%%%%%%%%%%%%%


%%%%%%%%%%%%%%%%%%%%%%%%%%%%%%%%%%%%%%%%%%%%%%%%%%%%%%%%%%%%%%%%%%%%%%

\section{TUJUAN SISTEM}

Tujuan dari dibangunnya sistem \textit{web services} ini adalah mempermudah pencatatan transaksi pembayaran yang terjadi melalui Bank agar tersimpan pada basis data Sistem Manajemen Informasi Objek Pajak PBB-P2.


\section{PEMODELAN SISTEM}

Sistem akan dimodelkan sebagai bentuk \textit{web services} yang menerima 3 (tiga) bentuk masukan, yaitu untuk melakukan \textit{inquiry}, pencatatan pembayaran, dan \textit{reversal}.

Karena perangkat pemrograman yang digunakan nantinya mendukung pemrograman berorientasi objek, maka akan lebih mudah apabila pendekatan pemodelan menggunakan \textit{Unified Modeling Language} (UML). Bentuk-bentuk diagram yang akan digunakan adalah sebagai berikut :

\begin{enumerate}
  \item Diagram \textit{Use-Case}
  
  Diagram ini akan mengilustrasikan gambaran utuh sebuah sistem yang berinteraksi dengan pengguna.
  
  \item Diagram \textit{Activity}
  
  Diagram ini akan mengilustrasikan aktifitas dari tiap objek yang saling berinteraksi membentuk sebuah sistem yang menerima masukkan, memprosesnya, dan kemudian menghasilkan sebuah keluaran yang dibutuhkan.
  
  \item Diagram \textit{Class}
  
  Diagram ini akan mengilustrasikan kelas-kelas pembentuk sistem berdasarkan objek-objek yang teridentifikasi sebelumnya.
  
  \item Diagram \textit{Sequence}
  
  Diagram ini akan mengilustrasikan alur interaksi dari tiap kelas 
  
\end{enumerate}




\section{AKTIVITAS PEMROSESAN DATA}


\end{document}